\input opmac
\chyph
\input lmfonts
\typosize[12/14]

\def\lc{\left\lfloor}   
\def\rc{\right\rfloor}

\chap Šifrování

\sec Substituční šifry

Šifrují podle vzorce:

$$
c = |a \cdot p + k|_m, 0 \leq c \leq m, gcd(a, m) = 1
$$

\noindent 
a dešifrují podle:
$$
p = |a^{-1} \cdot (c-b)|_m
$$

\sec Blokové šifry

Šifrují bloky velikosti $n$ podle vzorce:

$$
\left| \left(\matrix{c_1 \cr \vdots \cr c_n}\right) \right|_m = \left| \left(\matrix{A_{1,1} & \ldots & A_{1,n} \cr \vdots & \ddots & \vdots \cr A_{n_1} & \ldots & A_{n,n} \cr}\right) \cdot \left(\matrix{p_1 \cr \vdots \cr p_n}\right) \right|_m
$$

\noindent 
a dešifrují podle
$$
\left| \left(\matrix{p_1 \cr \vdots \cr p_n}\right) \right|_m = \left| \left(\matrix{A_{1,1} & \ldots & A_{1,n} \cr \vdots & \ddots & \vdots \cr A_{n_1} & \ldots & A_{n,n} \cr}\right) \cdot \left(\matrix{c_1 \cr \vdots \cr c_n}\right) \right|_m
$$

\noindent
\secc Příklad:

ST = STOP PAYMENT, velikost bloku = 3, 
$$
A = \left(\matrix{11 & 2 & 19 \cr 5 & 23 & 25 \cr 20 & 7 & 1 \cr}\right)
$$

\noindent
Rozdělíme ST do bloků: STO PPA YME NTX, kde X je padding. Poté pro každou trojici použijeme šifrovací matici $A$. První blok zašífrujeme takto:

$$
\left| A \cdot \left(\matrix{S \cr T \cr O}\right) \right|_{26} = \left| \left(\matrix{11 & 2 & 19 \cr 5 & 23 & 25 \cr 20 & 7 & 1 \cr}\right) \cdot \left(\matrix{18 \cr 19 \cr 14}\right) \right|_{26} = \left| \left(\matrix{11 \cdot 18 + 2 \cdot 19 + 19 \cdot 14 \cr 5 \cdot 18 + 23 \cdot 19 + 25 \cdot 14 \cr 20 \cdot 18 + 7 \cdot 19 + 1 \cdot 14}\right) \right|_{26} = \left(\matrix{8 \cr 19 \cr 13}\right)
$$

\noindent
A další bloky obdobně. Vyjde nám zpráva ITN NEP ACW ULA. Abychom tento text rozšifrovali, tak potřebujeme najít inverzí matici k matici $A$.

$$
det(A) = 11 \cdot 23 \cdot 1 + 2 \cdot 25 \cdot 20 + 19 \cdot 5 \cdot 7 - \left( 19 \cdot 23 \cdot 20 + 25 \cdot 7 \cdot 11 + 1 \cdot 2 \cdot 5 \right) = -8757
$$

\sec Exponenciální šífra

\begitems \style n
* Seskupíme čísla do $2 \cdot s$ dekadických číslic, kde $s$ je počet písmen v jednom bloku,
* tyto bloku tvoří čísla menší než $m$,
* zvolíme $2525 < m < 252525$, potom $s = 2$
* 

\enditems

\sec Zřízení společného klíče

\secc Pro 2 subjekty

\begitems \style n
* Zvolíme $m$ jako velké prvočíslo a $a$ jako celé číslo,
* Subjekt A si zvolí svoje tajné $k_1, gcd(k_1, m - 1) = 1$, vypočítá $y_1 = \left| a^{k_1} \right|_m$ a pošle subjektu B,
* Subjekt B si zvolí svoje tajné $k_2, gcd(k_2, m - 1) = 1$, vypočítá $y_2 = \left| a^{k_2} \right|_m$ a pošle subjektu A,
* Subjekt A spočítá společný klíč K = $\left|y_{2}^{k_1}\right|_m$,
* Subjekt B spočítá společný klíč K = $\left|y_{1}^{k_2}\right|_m$,
\enditems

\secc Pro N subjektů

\sec Proudové šifry

\chap Matematické postupy

\sec Modulární operace

\secc Dělení

Definice dělení jako násobení multiplikativní inverzí.

$$\left| 4 \over 3 \right|_{26} = |5 \cdot 3^{-1}|_{26} = |5 \cdot 9|_{26} = |45|_{26} = |9|_{26}$$

\sec Hledání inverze

Pro menší čísla lze použít tento vztah, kde $a$ a $a^{-1}$ jsou čísla k sobě navzájem iverzní v modulo $m$.
Inverze čísla $a$ v modulo $m$ existuje pouze pokud platí $gcd(a,m) = 1$

$$
|a^{-1}|_{m} \rightarrow a \cdot a^{-1} = 1 (mod 26)
$$
\noindent
Pro čísla větší lze použít rozšířený euklidův algoritmus.

\sec Modulo a velké mocniny

\sec Euklidův algoritmus

Používá se k nalezení největšího společného dělitele. $gcd(a,b) = d$, kde $a > b$

$$a_i \leftarrow a, b_i \leftarrow b$$
\centerline{Opakuj dokud není $c_i = 0$, pak $gcd(a,b) = b_i$}
$$a_i = k_i \cdot b_i + c_i$$
$$a_i = k_{i-1}$$
$$b_i = c_{i-1}$$

\noindent
Příklad:
$$gcd(130,15) = d$$
$$130 = 8 \cdot 15 + 10$$
$$15 = 1 \cdot 10 + 5$$
$$10 = 2 \cdot 5 \rightarrow gcd(130,15) = 5$$

\sec Rozšířený euklidův algoritmus

Používá se k nalezení inverze čísla $a$ v modulo $m$. $\alpha_{n-1} = a^{-1}$

\begmulti 2

\table{|r|ccc|}{\crli
m & 1 & 0 & - \cr
a & 0 & 1 & 1 \crli
$r_i$ & $\alpha_i$ & $\beta_i$ & $q_i$ \cr
$r_{i+1}$ & $\alpha_{i+1}$ & $\beta_{i+1}$ & $q_{i+1}$ \cr
$\vdots$ & $\vdots$ & $\vdots$ & $\vdots$ \cr
$r_{n} = 0$ & $\alpha_{n}$ & $\beta_{n}$ & $q_{n}$ 
\crli}

$$r_i = r_{i-2} - q_{i-1} \cdot r_{i-1}$$
$$\alpha_i = \alpha_{i-2} - q_{i-1} \cdot \alpha_{i-1}$$
$$\beta_i = \beta_{i-2} - q_{i-1} \cdot \beta_{i-1}$$
$$q_i = \lc r_{i-1} \over r_i \rc$$

\endmulti



\end