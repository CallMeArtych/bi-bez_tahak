\input opmac
\chyph
\input lmfonts
\typosize[12/14]

\def\lc{\left\lfloor}   
\def\rc{\right\rfloor}

\chap Šifrování

\sec Afinní šífrování

Šifrují podle vzorce:

$$
c = |a \cdot p + k|_m, 0 \leq c \leq m, gcd(a, m) = 1
$$

\noindent 
a dešifrují podle:
$$
p = |a^{-1} \cdot (c-b)|_m
$$

\sec Blokové šifry

Šifrují bloky velikosti $n$ podle vzorce:

$$
\left| \left(\matrix{c_1 \cr \vdots \cr c_n}\right) \right|_m = \left| \left(\matrix{A_{1,1} & \ldots & A_{1,n} \cr \vdots & \ddots & \vdots \cr A_{n_1} & \ldots & A_{n,n} \cr}\right) \cdot \left(\matrix{p_1 \cr \vdots \cr p_n}\right) \right|_m
$$

\noindent 
a dešifrují podle
$$
\left| \left(\matrix{p_1 \cr \vdots \cr p_n}\right) \right|_m = \left| \left(\matrix{A_{1,1} & \ldots & A_{1,n} \cr \vdots & \ddots & \vdots \cr A_{n_1} & \ldots & A_{n,n} \cr}\right) \cdot \left(\matrix{c_1 \cr \vdots \cr c_n}\right) \right|_m
$$

\noindent
\secc Příklad:

ST = STOP PAYMENT, velikost bloku = 3, 
$$
A = \left(\matrix{11 & 2 & 19 \cr 5 & 23 & 25 \cr 20 & 7 & 1 \cr}\right)
$$

\noindent
Rozdělíme ST do bloků: STO PPA YME NTX, kde X je padding. Poté pro každou trojici použijeme šifrovací matici $A$. První blok zašífrujeme takto:

$$
\left| A \cdot \left(\matrix{S \cr T \cr O}\right) \right|_{26} = \left| \left(\matrix{11 & 2 & 19 \cr 5 & 23 & 25 \cr 20 & 7 & 1 \cr}\right) \cdot \left(\matrix{18 \cr 19 \cr 14}\right) \right|_{26} = \left| \left(\matrix{11 \cdot 18 + 2 \cdot 19 + 19 \cdot 14 \cr 5 \cdot 18 + 23 \cdot 19 + 25 \cdot 14 \cr 20 \cdot 18 + 7 \cdot 19 + 1 \cdot 14}\right) \right|_{26} = \left(\matrix{8 \cr 19 \cr 13}\right)
$$

\noindent
A další bloky obdobně. Vyjde nám zpráva ITN NEP ACW ULA. Abychom tento text rozšifrovali, tak potřebujeme najít inverzí matici k matici $A$.

$$
det(A) = 11 \cdot 23 \cdot 1 + 2 \cdot 25 \cdot 20 + 19 \cdot 5 \cdot 7 - \left( 19 \cdot 23 \cdot 20 + 25 \cdot 7 \cdot 11 + 1 \cdot 2 \cdot 5 \right) = -8757
$$

\sec Exponenciální šífra

Pro většínu příkládů se uvažuje jako vstupní abeceda angličtina. Tím pádem $k = 25$ a $max = 2$.

\secc Šifrování

\begitems \style n
* Zvolíme $e$ a $m$, takové že $gcd(e,m) = 1$
* Převedeme každý znak v otevřeném textu na jeho číselnou reprezentaci zarovranou nulama na velikost největší číselné reprezentace znaku abecedy ($max$).
* Zvolíme velikost bloku $n$ takovou, aby platilo: $max \cdot n \cdot str(k) < m$ a zároveň byla největší možná.
* Vezmeme blok $n \cdot max$ číslic a použijeme na výsledné číslo vzorec:
$$c=\left| p^e \right|_{m}: e = \left| d^{-1} \right|_m$$
\enditems


\secc Dešifrování

\begitems \style n
* Vypočítáme velikost bloku číslic $n$.
* Použijeme vzorec pro dešifrování pro každý blok číslic velikosti $max \cdot n$ a převedeme po $max$ podblocích zpátky na znaky.
$$p=\left| c^d \right|_{m}: d = \left| e^{-1} \right|_m$$
\enditems

\secc Příklad

Mějme $m = 2633$ a nechť šifrovací klíč je $e = 29$. A otevřený text:
Jedná se o anglickou abecedu. Tím pádem $k = 25$ a $max = 2$.

$$THIS IS AN EXAMPLE OF AN EXPONENTIATION CIPHER$$

\begitems \style n
* Ověříme $gcd(e, m) = 1$.
* Určíme velikost bloku. $2525 < 2633$, tím pádem blokem budou 2 písmena převedená 4 číslice.
* Převedeme šifrový text na jeho číselný ekvivalent
$$
\matrix{
	1907 & 0818 & 0818 & 0013 & 0423 & 0012 & 1511 & 0414 & 0500 & 1304 \cr 
	2315 & 1413 & 0413 & 1908 & 0019 & 0814 & 1302 & 0815 & 0704 & 1723
}
$$
* Zašifrujeme každý blok podle vzorce.
$$
\matrix{
	2199 & 1745 & 1745 & 1209 & 2437 & 2425 & 1729 & 1619 & 0935 & 0960 \cr
	1072 & 1541 & 1701 & 1553 & 0735 & 2064 & 1351 & 1704 & 1741 & 1459
}
$$
\enditems

\noindent
Pro rozšifrování textu:

\begitems \style n
* Určíme velikost bloku číslic.
* Vypočítáme klíč pro dešifrování z šifrovacího.
* Na každý blok aplikujeme vzorec pro dešifrování.
$$
\matrix{
	1907 & 0818 & 0818 & 0013 & 0423 & 0012 & 1511 & 0414 & 0500 & 1304 \cr 
	2315 & 1413 & 0413 & 1908 & 0019 & 0814 & 1302 & 0815 & 0704 & 1723
}
$$
* Převedeme číslice po 2 zpátky na znaky.
$$THIS IS AN EXAMPLE OF AN EXPONENTIATION CIPHER$$
\enditems


\sec Zřízení společného klíče

\secc Pro 2 subjekty

\begitems \style n
* Zvolíme $m$ jako velké prvočíslo a $0 < a < m$.
* Subjekt A si zvolí svoje tajné $k_1, gcd(k_1, m - 1) = 1$, vypočítá $y_1 = \left| a^{k_1} \right|_m$ a pošle subjektu B.
* Subjekt B si zvolí svoje tajné $k_2, gcd(k_2, m - 1) = 1$, vypočítá $y_2 = \left| a^{k_2} \right|_m$ a pošle subjektu A.
* Subjekt A spočítá společný klíč $K = \left|a^{k_1 \cdot k_2}\right|_m = \left|y_{2}^{k_1}\right|_m$.
* Subjekt B spočítá společný klíč $K = \left|a^{k_1 \cdot k_2}\right|_m = \left|y_{1}^{k_2}\right|_m$.
\enditems

\secc Pro N subjektů

\begitems \style n
* Zvolíme $m$ jako velké prvočíslo a $0 < a < m$.
* Každý subjekt si zvolí svoje tajné $k_i, gcd(k_i, m - 1) = 1$ a vypočítá $y_i = \left| a^{k_i} \right|_m$.
* Klíč bude mít podobu $K = a^{k_1 \cdot \ldots \cdot k_n}$
* 
\enditems

\sec RSA

Bezpečtnostním prvkem je problém faktorizace.

\secc Příprava

\begitems \style n
* Subjekt A si zvolí 2 velká prvočísla $p$ a $q$.
* Subjekt A vypočítá modul $n = p \cdot q$.
* Subjekt A zvolí $e$ takové, že $gcd\left(e, \phi\left(n\right)\right) = 1$
* Subjekt A vypočítá $d = \left|e^{-1}\right|_{\phi\left(n\right)}$
* Subjekt A vypočítá veřejný klíč $VK = \left(n, e\right)$.
* Subjekt A vypočítá soukromý klíč $SK = \left(n, d\right)$.
* Subjekt A zveřejní $VK$, podle kterého nám budou ostatní subjekty posílat šifrové zprávy.
\enditems

\secc Šifrování

\begitems \style n
* Subjekt B si přečte veřejný klíč $VK = \left(n, e\right)$ subjektu A.
* Subjekt B převede zprávu do bloků délky řetězcové reprezentace $n$.
* Subjekt B zašífruje každý blok podle vztahu $c = \left|m^e\right|_n$
\enditems

\secc Dešifrování

\begitems \style n
* Subjekt A převede zprávu do bloků délky řetězcové reprezentace $n$.
* Subjekt A dešífrujeme každý blok podle vztahu $m = \left|c^d\right|_n$ za použití svého privátního klíče.
\enditems

\secc Příklad

$$p = 5, q = 11, e = 7$$
$$n = p \cdot q = 5 \cdot 11 = 55$$
$$VK = (n, e) = (55, 7)$$
$$d = \left|e^{-1}\right|_{\phi(n)} = \left|7^{-1}\right|_{\phi(55)} = \left|7^{-1}\right|_{\phi(5 \cdot 11)} = \left|7^{-1}\right|_{40} = 23$$
$$PK = (n, d) = (55, 23)$$

Šifrování
$$m = PUBLIC KEY CRYPTOGRAPHY$$
$$m = \matrix{1520 & 0111 & 0802 & 1004 & 2402 & 1724 & 1519 & 1406 & 1700 & 1507 & 2423}$$
$$c = \left|m_i^e\right|_n$$

Dešifrování
$$c = \ldots$$
$$m = \left|c_i^d\right|_n$$

\sec El Gamal

Založen na problému diskrétního logaritmu.

\secc Příprava

\begitems \style n
* Subjekt A si zvolí prvočísla $m$ a $g$ taková, aby $g$ bylo generátorem grupy $Z_m$.
* Subjekt A si náhodně zvolí náhodné $k_A$ takové, že $0 < k_A < m$.
* Subjekt A vypočítá $y_A = \left|g^{k_A}\right|_m$
* Subjekt A nasdílí $VK = (m, g, y_A)$.
* Subjekt A si uloží $SK = (k_A)$.
\enditems

\secc Šifrování

\begitems \style n
* Subjekt B si náhodně zvolí náhodné $k_B$ takové, že $0 < k_B < m$.
* Subjekt B vypočítá společný sdílený klíč $K = \left|g^{k_A \cdot k_B}\right|_m = \left|y_A^{k_B}\right|_m$
* Subjekt B zašifruje zprávu podle vztahu $c = \left|p \cdot K\right|_m$.
* Subjekt B vypočítá $y_B = \left|g^{k_B}\right|_m$
* Subjekt B pošle subjektu A dvojici $\left(y_B, c\right)$
\enditems

\secc Dešifrování

\begitems \style n
* Subjekt A dostal od B dvojici $\left(y_B, c\right)$.
* Subjekt A vypočítá společný sdílený klíč $K = \left|g^{k_A \cdot k_B}\right| = \left|y_B^{k_A}\right|$
* Subjekt A dešifruje zprávu podle $p = \left|c \cdot K^{-1}\right|_m$
\enditems

\sec Operační módy

\centerline {\picwidth=\hsize \inspic operacni_mody.png }

\secc EBC - Electronic CodeBook

Stejné bloky OT mají stejné bloky ŠT.

$$ŠT_i = E_k\left(OT_i\right)$$
$$OT_i = D_k\left(ŠT_i\right)$$

\secc CBC - Cipher Block Chaining

Možnost samosynchronizace při chybě (pokazí - pokažené - OK). Inicilizační vektor nám zajistí, že 2 stejné zprávy mohou být odlišně zašifrovány.

$$ŠT_0 = IV$$
$$ŠT_i = E_k\left(OT_i \oplus ŠT_{i-1} \right)$$
$$OT_i = D_k\left(ŠT_{i} \right) \oplus ŠT_{i-1}$$

\secc CFB - Cipher FeedBack

Čistě synchronní.

$$ŠT_0 = IV$$
$$ŠT_i = E_k\left(ŠT_{i-1} \right) \oplus OT_i$$
$$OT_i = E_k\left(ŠT_{i-1} \right) \oplus ŠT_i$$

\secc OFB - Output FeedBack

Čistě synchronní.

$$H = IV = ŠT_0$$
$$ŠT_i = OT_i \oplus H, H = E_k\left(H\right)$$
$$OT_i = ŠT_i \oplus H, H = E_k\left(H\right)$$

\sec Narozeninový paradox

\sec Entropie

Množství informace obsažené ve zprávě. 
Entropie zprávy ze zdroje $X$, kde $p_1 \ldots p_n$ jsou pravděpodobnosti zpráv $X_1 \ldots X_n$ zdroje $X$.

$$
H(X) = -\sum_{i = 1}^{n}{p_i \cdot \log_{2}{(p_i)}}
$$
\noindent
$H(X)$ vyjadřuje počet bitů nutných k zakódování zprávy $X_i$

\begitems \style o
* Maximalni entropie nastává pokud všechný generované zprávy maji stejnou pravděpodobnost $p_1 = p_2 = \ldots = p_n$.
Potom $H(X) = \log_{2}{(n)}$.
* Minimalní entropie nastává pokud je generovaná jedina zpráva s pravděpodobnosti $p_1 = 1$.
Potom $H(X) = 0$.
\enditems

\secc Pomůcky

$$
-\log_{2}{ \left( {1 \over 2^n} \right) }
= -(-\log_{2}{(2^n)})
= -(-n)
= n
$$

\secc Příklad

\table{|c|c|c|}{\crli
$n$ & $P(n)$ & $p_i$ \crli
4 & $1 / 16$ & $1 / 4$ \cr
2 & $1 / 8$ & $1 / 4$ \cr
2 & $1 / 4$ & $1 / 2$ \cr
\crli}

\begitems \style o
* $n$ - počet zpráv jednoho typu.
* $P(n)$ - pravděpodobnost výskytu jedné zprávy z $n$.
* $p_i$ - celková pravděpodobnost výskytu zprávý jednoho typu ($P(n) \cdot n$).
\enditems

$$ \displaylines{
H(X) = - \sum_{i = 1}^{n}{ p_i \cdot \log_{2}{(p_i)} }
= - {1 \over 4} \cdot \log_{2}{ \left( {1 \over 4} \right) }
- {1 \over 4} \cdot \log_{2}{ \left( {1 \over 4} \right) }
- {1 \over 2} \cdot \log_{2}{ \left( {1 \over 2} \right) } \cr
= {2 \over 4} + {2 \over 4} + {1 \over 2}
= {6 \over 4} = 1.5
}
$$

\secc Příklad

Zdroj zpráv X posíla 4 různé zprávy s pravděpodobnostmi $p_1 = 1/8$, $p_2 = 1/4$, $p_3 = 1/2$, $p_4 = 1/8$

$$ \displaylines{
H(X) = - \sum_{i = 1}^{n}{ p_i \cdot \log_{2}{(p_i)} }
= - {1 \over 8} \cdot \log_{2}{ \left( {1 \over 8} \right) }
- {1 \over 4} \cdot \log_{2}{ \left( {1 \over 4} \right) }
- {1 \over 2} \cdot \log_{2}{ \left( {1 \over 2} \right) }
- {1 \over 8} \cdot \log_{2}{ \left( {1 \over 8} \right) } \cr
= {3 \over 8} + {2 \over 4} + {1 \over 2} + {3 \over 8}
= {14 \over 8} = 1.75
}
$$
\noindent
Jaká by byla entropie, pokud bychom změnili pravděpodobnosti tak, aby byly všechny stejné.

$$
H(X) = - \sum_{i = 1}^{n}{ p_i \cdot \log_{2}{(p_i)} }
= -4 \cdot {1 \over 4} \cdot \log_{2}{ \left( {1 \over 4} \right) }
= 4 \cdot {2 \over 4} = 2
$$

\noindent
Neboli $H(X) = \log_{2}{(n)} = \log_{2}{(4)} = 2$.

\sec Vzdálenost jednoznačnosti

Počet znaků OT, pro které množství informace o OT obsažené v ŠT dosáhne takového bodu, že je možný jen jediný OT.

\chap Matematické postupy

\sec Modulární operace

\secc Dělení

Definice dělení jako násobení multiplikativní inverzí.

$$\left| 4 \over 3 \right|_{26} = |5 \cdot 3^{-1}|_{26} = |5 \cdot 9|_{26} = |45|_{26} = |9|_{26}$$

\sec Násobení matic

Násobíme řádek krát sloupec po složkách.

$$
\left(\matrix{1 & 2 & 3 \cr 4 & 5 & 6}\right) \cdot \left(\matrix{1 & 2 \cr 3 & 4 \cr 5 & 6}\right) = \left(\matrix{1 \cdot 1 + 2 \cdot 3 + 3 \cdot 5 & 1 \cdot 2 + 2 \cdot 4 + 3 \cdot 6 \cr 4 \cdot 1 + 5 \cdot 3 + 6 \cdot 5 & 4 \cdot 2 + 5 \cdot 4 + 6 \cdot 6}\right) = \left(\matrix{22 & 28 \cr 49 & 64}\right)
$$

\sec Hledání inverzní matice

\sec Hledání inverze

Pro menší čísla lze použít tento vztah, kde $a$ a $a^{-1}$ jsou čísla k sobě navzájem iverzní v modulo $m$.
Inverze čísla $a$ v modulo $m$ existuje pouze pokud platí $gcd(a,m) = 1$

$$
|a^{-1}|_{m} \rightarrow a \cdot a^{-1} = 1 (mod 26)
$$
\noindent
Pro čísla větší lze použít rozšířený euklidův algoritmus.

\sec Modulo a velké mocniny

Mějme zadáno $\left|a^k\right|_m$, kde $k$ je opravdu velké číslo.

\begitems \style n
* Rozdělíme číslo na součin jeho mocnin: $a^k=a^{k_1} \ldots \cdot a^{k_n}$, kde $\sum k_n = k$ a $k_n$ jsou čísla ve tvaru $2 \cdot n \in Z$
* pro každé číslo $i < max(k_n)$ množiny mocnin vypočítáme jeho hodnotu v modulu m, využíváme k tomu předchozí hodnoty.
\enditems

$$\left|5^{28}\right|_{17} = \left|5^{16} \cdot 5^8 \cdot 5^4\right| = \left|1 \cdot -1 \cdot -4\right|_{17} = 4$$
$$5^{28} = 5^{16} \cdot 5^8 \cdot 5^4$$
$$\left|5^1\right|_{17} = \left|5 \right|_{17} = \left|5\right|_{17}$$
$$\left|5^2\right|_{17} = \left|25 \right|_{17} = \left|8\right|_{17}$$
$$\left|5^4\right|_{17} = \left|5^2 \cdot 5^2\right|_{17} = \left|8 \cdot 8\right|_{17} = \left|13\right|_{17} = \left|-4\right|_{17}$$
$$\left|5^8\right|_{17} = \left|5^4 \cdot 5^4\right|_{17} = \left|-4 \cdot -4\right|_{17} = \left|16\right|_{17} = \left|-1\right|_{17}$$
$$\left|5^{16}\right|_{17} = \left|5^8 \cdot 5^8\right|_{17} = \left|-1 \cdot -1\right|_{17} = \left|1\right|_{17}$$

\sec Euklidův algoritmus

Používá se k nalezení největšího společného dělitele. $gcd(a,b) = d$, kde $a > b$

$$a_i \leftarrow a, b_i \leftarrow b$$
\centerline{Opakuj dokud není $c_i = 0$, pak $gcd(a,b) = b_i$}
$$a_i = k_i \cdot b_i + c_i$$
$$a_i = k_{i-1}$$
$$b_i = c_{i-1}$$

\noindent
Příklad:
$$gcd(130,15) = d$$
$$130 = 8 \cdot 15 + 10$$
$$15 = 1 \cdot 10 + 5$$
$$10 = 2 \cdot 5 \rightarrow gcd(130,15) = 5$$

\sec Rozšířený euklidův algoritmus

Používá se k nalezení inverze čísla $a$ v modulo $m$. $\beta_{n-1} = a^{-1}$

\begmulti 2

\table{|r|ccc|}{\crli
m & 1 & 0 & - \cr
a & 0 & 1 & $q_i - 1$ \crli
$r_i$ & $\alpha_i$ & $\beta_i$ & $q_i$ \cr
$r_{i+1}$ & $\alpha_{i+1}$ & $\beta_{i+1}$ & $q_{i+1}$ \cr
$\vdots$ & $\vdots$ & $\vdots$ & $\vdots$ \cr
$r_{n} = 0$ & $\alpha_{n}$ & $\beta_{n}$ & $q_{n}$ 
\crli}

$$r_i = r_{i-2} - q_{i-1} \cdot r_{i-1}$$
$$\alpha_i = \alpha_{i-2} - q_{i-1} \cdot \alpha_{i-1}$$
$$\beta_i = \beta_{i-2} - q_{i-1} \cdot \beta_{i-1}$$
$$q_i = \lc r_{i-1} \over r_i \rc$$

\endmulti

\sec Multiplikativní grupy

Algoritmus ověření zda $g$ je generátorem grupy $m$.

$$\left|g^i\right| = Z_m; i = 1 \ldots m - 1$$

\noindent
Tvoříme řádky v tabulce. Pokud narazíme na $z = 1$, pak to znamená, že všechny dosud napsaná $z$ se zopakují (včetně té 1), takže se začínáme cyklit. Pokud narazíme na $z = -1$, tak se začne všechno opakovat jen s opačným znaménkem.

\table{|c|c|}{\crli
$i$ & $\left|g^i\right|_m$ \crli
1 & $z_1$ \cr
2 & $z_2$ \cr
$\vdots$ & $\vdots$ \cr
$m-1$ & $z_{m-1}$ \cr
\crli}

\chap Hašovací funkce

Hašovací funkce $h$ zpracovává praktický neomezeně dlouhá vstupní data $M$ na výstupní hašový kód $h(M)$ stanovené délky $n$ bitů.
Kvůli potenciálně strojově nezvladnutelné délce vstupní zprávy, ta musí být nejprve rozdělená na bloky konstantní velikosti (např. 512 bitů u MD5 a SHA-0/1/256).

Ke vstupní zprávě je vždy doplněn padding tak, že celá zpráva je zarovnáná na celé bloky (i přesto že byla zarovnáná i bez paddingu). Padding začíná 1 a je doplněn nulama tak, aby do konce bloku zbývalo 64 bitu, kam je uložená délka původní zprávy.

\begitems \style o
* MD5 (128-bitový hash kód)
* SHA-1 (160-bitový hash kód)
* SHA-256 (256-bitový hash kód)
* SHA-512 (512-bitový hash kód)
\enditems

% {\bf Damgard-Merklova konstrukce}
% \begitems \style n
% * Hašování probíha postupně po blocích $M_i$ pomocí tzv. kompresní funkce.
% * Výsledkem kompresní funkce v každem kroku $i$ je tzv. kontext $H_{i}$.
% * Do kompresní funkce vždy vstupuje příslušný blok $M_i$ a kontext $H_{i-1}$ z předchozího kroku.
% * $H_0$ je inicializační vektor $IV$.
% * Celý poslední kontext $H_N$ nebo jeho čast (závisí na délce haš kódu zvolené metody a velikosti kontextu) je výsledný haš kód.

% Pro DM konstrukcí plátí -- pokud je kompresní funkce bezkolizní, pak je cela hašovací funkce z ní konstruovaná je bezkolizní.

\sec Kolize 1. řádu

Nalezení dvou různých zprav se stejným haš kódem.

$$
P(N,k) = 1 - \prod^{k-1}_{i=1}\left( 1 - {i \over N} \right)
\approx 1 - e^{-{k(k-1) \over 2N}}= p
$$

\begitems \style o
* $p$ je pravděpodobnost nalezení kolize.
* $N$ je počet všech možných haš kódů.
* $k$ je počet haš kódů postačujicí k nalezení kolize s pravděpodobností $p$.
\enditems

$$k \approx \sqrt{-2N\ln(1 - p)}$$

\bigskip\centerline
{\bf Pro SHA-1 (160-bitový hash kód) a pravděpodobnosti 50\%}

$$N = 2^{160} = 2^n, p = 0.5$$
$$
k \approx \sqrt{-2 \ln(1 - 0.5) 2^n} \approx \sqrt{-2 \cdot -0.7}\sqrt{2^n}
\approx \sqrt{1.4} \cdot 2^{n \over 2} \approx 1.2 \cdot 2^{n \over 2}
\approx 2^{160 \over 2} \approx 2^{80}
$$

\bigskip\centerline
{\bf Pro SHA-512 (512-bitový hash kód) a pravděpodobnosti 70\%}

$$N = 2^{512} = 2^n, p = 0.7$$
$$
k \approx \sqrt{-2 \ln(1 - 0.7) 2^n} \approx \sqrt{-2 \cdot -1.2}\sqrt{2^n}
\approx \sqrt{2.4} \cdot 2^{n \over 2} \approx 1.55 \cdot 2^{n \over 2}
\approx 2^{512 \over 2} \approx 2^{256}
$$

\bigskip\centerline
{\bf Pro SHA-256 (256-bitový hash kód) a pravděpodobnosti 99\%}

$$N = 2^{256} = 2^n, p = 0.99$$
$$
k \approx \sqrt{-2 \ln(1 - 0.99) 2^n} \approx \sqrt{-2 \cdot -4.6}\sqrt{2^n}
\approx \sqrt{9.2} \cdot 2^{n \over 2} \approx 3 \cdot 2^{n \over 2}
\approx 2^{256 \over 2} \approx 2^{128}
$$

\centerline{Vypada, že vzorec neplati pro pravděpodobnost jinou nez 50\%.}

\bigskip\centerline
{\bf Pro SHA-1 (160-bitový hash kód) a pravděpodobnosti 100\%}

$$N = 2^{160} = 2^n, p = 1$$
$$
k \approx \sqrt{-2 \ln(1 - 1) 2^n} = \sqrt{-2 \lim_{x \rightarrow 0} {(\ln(x))} 2^n}
= \sqrt{-2 \cdot -\infty \cdot 2^n} = \infty
$$

\centerline{Logický $k$ nemůže být vetší než $N$, proto $k = N = 2^{160}$}

\bigskip\centerline
{\bf Pro SHA-1 (160-bitový hash kód) a pravděpodobnosti 0\%}

$$N = 2^{160} = 2^n, p = 0$$
$$
k \approx \sqrt{-2 \ln(1 - 0) 2^n} = \sqrt{-2 \ln(1) 2^n}
= \sqrt{-2 \cdot 0 \cdot 2^n} = 0
$$

\centerline{Logický může platit i $k = 1$.}

\sec Kolize 2. řádu

Nalezení druhé zprávy se stejným haš kódem k zadané zprávě.

$$P(N,k) = 1 - \left( 1 - {1 \over N} \right)^k = p$$
$$k = {\ln(1 - p) \over \ln\left(1 - {1 \over N}\right)}$$

\centerline{Pro malá $x$ platí $\ln(1 + x) \approx x$.}

$$
k \approx {\ln(1 -p) \over -{1 \over N}} = N \ln\left({1 \over 1-p}\right)
$$

\bigskip\centerline
{\bf Pro SHA-1 (160-bitový hash kód) a pravděpodobnosti 50\%}

$$N = 2^{160} = 2^n, p = 0.5$$
$$
k \approx 2^n \ln\left({1 \over {1 \over 2}}\right) = 2^n \ln 2 \approx 2^n = 2^{160}
$$

\bigskip\centerline
{\bf Pro SHA-1 (160-bitový hash kód) a pravděpodobnosti 0\%}

$$N = 2^{160} = 2^n, p = 0$$
$$
k \approx 2^n \ln\left({1 \over {1 \over 1}}\right) = 2^n \ln 1 \approx 2^n \cdot 0 = 0
$$

\bigskip\centerline
{\bf Pro SHA-1 (160-bitový hash kód) a pravděpodobnosti 100\%}

$$N = 2^{160} = 2^n, p = 1$$
$$
k \approx 2^n \lim_{x \rightarrow 0}{\ln\left({1 \over x}\right)}
= 2^n \infty = \infty
$$

\centerline{Logický $k$ nemůže být vetší než $N$, proto $k = N = 2^{160}$}

\end
