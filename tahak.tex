\input opmac
\chyph
\input lmfonts
\typosize[12/14]

\def\lc{\left\lfloor}   
\def\rc{\right\rfloor}

\chap Šifrování

\sec Afinní šífrování

Šifrují podle vzorce:

$$
c = |a \cdot p + k|_m, 0 \leq c \leq m, gcd(a, m) = 1
$$

\noindent 
a dešifrují podle:
$$
p = |a^{-1} \cdot (c-b)|_m
$$

\sec Blokové šifry

Šifrují bloky velikosti $n$ podle vzorce:

$$
\left| \left(\matrix{c_1 \cr \vdots \cr c_n}\right) \right|_m = \left| \left(\matrix{A_{1,1} & \ldots & A_{1,n} \cr \vdots & \ddots & \vdots \cr A_{n_1} & \ldots & A_{n,n} \cr}\right) \cdot \left(\matrix{p_1 \cr \vdots \cr p_n}\right) \right|_m
$$

\noindent 
a dešifrují podle
$$
\left| \left(\matrix{p_1 \cr \vdots \cr p_n}\right) \right|_m = \left| \left(\matrix{A_{1,1} & \ldots & A_{1,n} \cr \vdots & \ddots & \vdots \cr A_{n_1} & \ldots & A_{n,n} \cr}\right) \cdot \left(\matrix{c_1 \cr \vdots \cr c_n}\right) \right|_m
$$

\noindent
\secc Příklad:

ST = STOP PAYMENT, velikost bloku = 3, 
$$
A = \left(\matrix{11 & 2 & 19 \cr 5 & 23 & 25 \cr 20 & 7 & 1 \cr}\right)
$$

\noindent
Rozdělíme ST do bloků: STO PPA YME NTX, kde X je padding. Poté pro každou trojici použijeme šifrovací matici $A$. První blok zašífrujeme takto:

$$
\left| A \cdot \left(\matrix{S \cr T \cr O}\right) \right|_{26} = \left| \left(\matrix{11 & 2 & 19 \cr 5 & 23 & 25 \cr 20 & 7 & 1 \cr}\right) \cdot \left(\matrix{18 \cr 19 \cr 14}\right) \right|_{26} = \left| \left(\matrix{11 \cdot 18 + 2 \cdot 19 + 19 \cdot 14 \cr 5 \cdot 18 + 23 \cdot 19 + 25 \cdot 14 \cr 20 \cdot 18 + 7 \cdot 19 + 1 \cdot 14}\right) \right|_{26} = \left(\matrix{8 \cr 19 \cr 13}\right)
$$

\noindent
A další bloky obdobně. Vyjde nám zpráva ITN NEP ACW ULA. Abychom tento text rozšifrovali, tak potřebujeme najít inverzí matici k matici $A$.

$$
det(A) = 11 \cdot 23 \cdot 1 + 2 \cdot 25 \cdot 20 + 19 \cdot 5 \cdot 7 - \left( 19 \cdot 23 \cdot 20 + 25 \cdot 7 \cdot 11 + 1 \cdot 2 \cdot 5 \right) = -8757
$$

\sec Exponenciální šífra

Pro většínu příkládů se uvažuje jako vstupní abeceda angličtina. Tím pádem $k = 25$ a $max = 2$.

\secc Šifrování

\begitems \style n
* Zvolíme $e$ a $m$, takové že $gcd(e,m) = 1$
* Převedeme každý znak v otevřeném textu na jeho číselnou reprezentaci zarovranou nulama na velikost největší číselné reprezentace znaku abecedy ($max$).
* Zvolíme velikost bloku $n$ takovou, aby platilo: $max \cdot n \cdot str(k) < m$ a zároveň byla největší možná.
* Vezmeme blok $n \cdot max$ číslic a použijeme na výsledné číslo vzorec:
$$c=\left| p^e \right|_{m}: e = \left| d^{-1} \right|_m$$
\enditems


\secc Dešifrování

\begitems \style n
* Vypočítáme velikost bloku číslic $n$.
* Použijeme vzorec pro dešifrování pro každý blok číslic velikosti $max \cdot n$ a převedeme po $max$ podblocích zpátky na znaky.
$$p=\left| c^d \right|_{m}: d = \left| e^{-1} \right|_m$$
\enditems

\secc Příklad

Mějme $m = 2633$ a nechť šifrovací klíč je $e = 29$. A otevřený text:
Jedná se o anglickou abecedu. Tím pádem $k = 25$ a $max = 2$.

$$THIS IS AN EXAMPLE OF AN EXPONENTIATION CIPHER$$

\begitems \style n
* Ověříme $gcd(e, m) = 1$.
* Určíme velikost bloku. $2525 < 2633$, tím pádem blokem budou 2 písmena převedená 4 číslice.
* Převedeme šifrový text na jeho číselný ekvivalent
$$
1907\ 0818\ 0818\ 0013\ 0423\ 0012\ 1511\ 0414\ 0500\ 1304
2315\ 1413\ 0413\ 1908\ 0019\ 0814\ 1302\ 0815\ 0704\ 1723
$$
* Zašifrujeme každý blok podle vzorce.
$$
2199\ 1745\ 1745\ 1209\ 2437\ 2425\ 1729\ 1619\ 0935\ 0960
1072\ 1541\ 1701\ 1553\ 0735\ 2064\ 1351\ 1704\ 1741\ 1459
$$
\enditems

\noindent
Pro rozšifrování textu:

\begitems \style n
* Určíme velikost bloku číslic.
* Vypočítáme klíč pro dešifrování z šifrovacího.
* Na každý blok aplikujeme vzorec pro dešifrování.
$$
1907\ 0818\ 0818\ 0013\ 0423\ 0012\ 1511\ 0414\ 0500\ 1304
2315\ 1413\ 0413\ 1908\ 0019\ 0814\ 1302\ 0815\ 0704\ 1723
$$
* Převedeme číslice po 2 zpátky na znaky.
$$THIS IS AN EXAMPLE OF AN EXPONENTIATION CIPHER$$
\enditems


\sec Zřízení společného klíče

\secc Pro 2 subjekty

\begitems \style n
* Zvolíme $m$ jako velké prvočíslo a $0 < a < m$.
* Subjekt A si zvolí svoje tajné $k_1, gcd(k_1, m - 1) = 1$, vypočítá $y_1 = \left| a^{k_1} \right|_m$ a pošle subjektu B.
* Subjekt B si zvolí svoje tajné $k_2, gcd(k_2, m - 1) = 1$, vypočítá $y_2 = \left| a^{k_2} \right|_m$ a pošle subjektu A.
* Subjekt A spočítá společný klíč $K = \left|a^{k_1 \cdot k_2}\right|_m = \left|y_{2}^{k_1}\right|_m$.
* Subjekt B spočítá společný klíč $K = \left|a^{k_1 \cdot k_2}\right|_m = \left|y_{1}^{k_2}\right|_m$.
\enditems

\secc Pro N subjektů

\begitems \style n
* Zvolíme $m$ jako velké prvočíslo a $0 < a < m$.
* Každý subjekt si zvolí svoje tajné $k_i, gcd(k_i, m - 1) = 1$ a vypočítá $y_i = \left| a^{k_i} \right|_m$.
* Klíč bude mít podobu $K = a^{k_1 \cdot \ldots \cdot k_n}$
* 
\enditems

\sec Proudové šifry

\chap Matematické postupy

\sec Modulární operace

\secc Dělení

Definice dělení jako násobení multiplikativní inverzí.

$$\left| 4 \over 3 \right|_{26} = |5 \cdot 3^{-1}|_{26} = |5 \cdot 9|_{26} = |45|_{26} = |9|_{26}$$

\sec Násobení matic

Násobíme řádek krát sloupec po složkách.

$$
\left(\matrix{1 & 2 & 3 \cr 4 & 5 & 6}\right) \cdot \left(\matrix{1 & 2 \cr 3 & 4 \cr 5 & 6}\right) = \left(\matrix{1 \cdot 1 + 2 \cdot 3 + 3 \cdot 5 & 1 \cdot 2 + 2 \cdot 4 + 3 \cdot 6 \cr 4 \cdot 1 + 5 \cdot 3 + 6 \cdot 5 & 4 \cdot 2 + 5 \cdot 4 + 6 \cdot 6}\right) = \left(\matrix{22 & 28 \cr 49 & 64}\right)
$$

\sec Hledání inverzní matice

\sec Hledání inverze

Pro menší čísla lze použít tento vztah, kde $a$ a $a^{-1}$ jsou čísla k sobě navzájem iverzní v modulo $m$.
Inverze čísla $a$ v modulo $m$ existuje pouze pokud platí $gcd(a,m) = 1$

$$
|a^{-1}|_{m} \rightarrow a \cdot a^{-1} = 1 (mod 26)
$$
\noindent
Pro čísla větší lze použít rozšířený euklidův algoritmus.

\sec Modulo a velké mocniny

Mějme zadáno $\left|a^k\right|_m$, kde $k$ je opravdu velké číslo.

\begitems \style n
* Rozdělíme číslo na součin jeho mocnin: $a^k=a^{k_1} \ldots \cdot a^{k_n}$, kde $\sum k_n = k$ a $k_n$ jsou čísla ve tvaru $2 \cdot n \in Z$
* pro každé číslo $i < max(k_n)$ množiny mocnin vypočítáme jeho hodnotu v modulu m, využíváme k tomu předchozí hodnoty.
\enditems

$$\left|5^{28}\right|_{17} = \left|5^{16} \cdot 5^8 \cdot 5^4\right| = \left|1 \cdot -1 \cdot -4\right|_{17} = 4$$
$$5^{28} = 5^{16} \cdot 5^8 \cdot 5^4$$
$$\left|5^1\right|_{17} = \left|5 \right|_{17} = \left|5\right|_{17}$$
$$\left|5^2\right|_{17} = \left|25 \right|_{17} = \left|8\right|_{17}$$
$$\left|5^4\right|_{17} = \left|5^2 \cdot 5^2\right|_{17} = \left|8 \cdot 8\right|_{17} = \left|13\right|_{17} = \left|-4\right|_{17}$$
$$\left|5^8\right|_{17} = \left|5^4 \cdot 5^4\right|_{17} = \left|-4 \cdot -4\right|_{17} = \left|16\right|_{17} = \left|-1\right|_{17}$$
$$\left|5^{16}\right|_{17} = \left|5^8 \cdot 5^8\right|_{17} = \left|-1 \cdot -1\right|_{17} = \left|1\right|_{17}$$

\sec Euklidův algoritmus

Používá se k nalezení největšího společného dělitele. $gcd(a,b) = d$, kde $a > b$

$$a_i \leftarrow a, b_i \leftarrow b$$
\centerline{Opakuj dokud není $c_i = 0$, pak $gcd(a,b) = b_i$}
$$a_i = k_i \cdot b_i + c_i$$
$$a_i = k_{i-1}$$
$$b_i = c_{i-1}$$

\noindent
Příklad:
$$gcd(130,15) = d$$
$$130 = 8 \cdot 15 + 10$$
$$15 = 1 \cdot 10 + 5$$
$$10 = 2 \cdot 5 \rightarrow gcd(130,15) = 5$$

\sec Rozšířený euklidův algoritmus

Používá se k nalezení inverze čísla $a$ v modulo $m$. $\beta_{n-1} = a^{-1}$

\begmulti 2

\table{|r|ccc|}{\crli
m & 1 & 0 & - \cr
a & 0 & 1 & $q_i - 1$ \crli
$r_i$ & $\alpha_i$ & $\beta_i$ & $q_i$ \cr
$r_{i+1}$ & $\alpha_{i+1}$ & $\beta_{i+1}$ & $q_{i+1}$ \cr
$\vdots$ & $\vdots$ & $\vdots$ & $\vdots$ \cr
$r_{n} = 0$ & $\alpha_{n}$ & $\beta_{n}$ & $q_{n}$ 
\crli}

$$r_i = r_{i-2} - q_{i-1} \cdot r_{i-1}$$
$$\alpha_i = \alpha_{i-2} - q_{i-1} \cdot \alpha_{i-1}$$
$$\beta_i = \beta_{i-2} - q_{i-1} \cdot \beta_{i-1}$$
$$q_i = \lc r_{i-1} \over r_i \rc$$

\endmulti



\end